\section{Conclusion}

The present paper presents an expert assessment on co-simulation, taking on the social and empirical aspect, with a focus on promising standards and tools, current challenges and research needs. 
As a methodological foundation of this study, the Delphi method was adopted.
Furthermore, a quantitative analysis of the SWOT of co-simulation utilizing the Analytic Hierarchy Process has been conducted.
The authors consider the following findings from the empirical data as the most important:

\begin{compactitem}
\item Experts consider FMI as the most promising standards for continuous time, discrete event  and  hybrid  co-simulation;
\item Experts frequently have difficulties in practical aspects, like IT-prerequisites in cross-company collaboration and difficulties due to
insufficient communication between theorists and practitioners;
\item Research needs xxxxxx; and 
\item The  results  of  the  SWOT-AHP  analysis  indicate that strengths and opportunities factors predominate.  The factor with the  highest  score  is  the  external  opportunity  of  User-friendly  tools  including pre-defined master algorithms, integrated error estimation, etc.;
\end{compactitem} \leavevmode
\\
It is our hope that the results of this study increase transparency and facilitate a structured development of co-simulation standards and tools.





